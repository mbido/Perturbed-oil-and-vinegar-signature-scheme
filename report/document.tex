\documentclass[12pt, a4paper]{article}

\usepackage{amsfonts}
\usepackage{amsmath}
\usepackage{amssymb}
\usepackage{amsthm}
\usepackage{graphicx}
\usepackage{hyperref}
\usepackage{float}
\usepackage{lipsum} % Pour ajouter du texte dans l'exemple
\usepackage{fancyhdr} % Pour personnaliser les en-têtes et pieds de page

% \hypersetup{
	%     colorlinks=true,
	%     linkcolor=blue,
	%     filecolor=magenta,      
	%     urlcolor=cyan,
	%     citecolor=blue,
	% }

% header & footer
\pagestyle{fancy}
\fancyhf{}
\renewcommand{\headrulewidth}{0.4pt}
\renewcommand{\footrulewidth}{0.4pt}
\fancyhead[L]{\textit{CCA -- Research project}}
\fancyhead[R]{\thepage}

\begin{document}
	
	
	\begin{titlepage}
		\centering
		{\Huge\textbf{CCA Project -- Unbalanced Oil and Vinegar Signature Schemes}\par}
		\vspace{0.5cm}
		{\Large Nicholas Bell and Matthieu Bidault\par}
		\vspace{0.5cm}
		{\Large supervised by Pierre Pébereau\par}
		\vfill
		\textit{2024-2025}
		\vspace*{\fill}
	\end{titlepage}
	
	\setcounter{tocdepth}{3}
	\tableofcontents
	\thispagestyle{fancy}
	\newpage
	
	\section{Oil and Vinegar Signatures}
	
	\subsection{Schema Description}
	
	We begin by briefly presenting Kipnis and Shamir's variant of the Oil and Vinegar signature schema. Let $(m_1,\dots,m_k) \in \mathbb{F}_q$ be our message.
	
	\subsubsection{Key Generation}
	
	Our \textbf{private key} $A$ is a randomly chosen invertible matrix in $\mathbb{F}_q^{2k\times2k}$.\\
	
	For our public key we first generate $k$ random matrices $F_1,\dots,F_k \in \mathbb{F}_q^{2k\times2k}$, such that the upper-left quadrant of each $F_i$ is zero, i.e. $$F_i = \begin{bmatrix}
	0 & B_1\\
	B_2 & B_3
	\end{bmatrix}\text{, for }B_i \in \mathbb{F}_q^{k\times k}$$. Our \textbf{public key} is $G_1,\dots,G_k$, with $$G_i := A^TF_iA$$.
	
	\subsubsection{Signature}
	
	Given a message $m \in \mathbb{F}_q^{k}$, we first 
	
	Étant donné un message $M \in \mathcal{F}_n^k$, nous voulons construire une \textbf{signature} $X = (x_1, \dots, x_{2k}) \in \mathcal{F}_n^{2k}$, c'est-à-dire :
	
	$$
	\begin{cases}
		G_1(x_1,...,x_{2k}) = m_1 \\
		G_2(x_1,...,x_{2k}) = m_2 \\
		\vdots \\
		G_k(x_1,...,x_{2k}) = m_k
	\end{cases}
	$$
	
	Pour cela, nous créons d'abord $Y = (y_1, \dots, y_{2k}) \in \mathcal{F}_n^{2k}$. La première partie $(y_1, \dots, y_k)$ est appelée l'\textbf{huile} et la seconde partie $(y_{k + 1}, \dots, y_{2k})$ est appelée le \textbf{vinaigre}. Le vinaigre est généré aléatoirement. Pour obtenir l'huile, nous devons résoudre le système suivant :
	
	$$
	\begin{cases}
		Y^\top F_1 Y = m_1 \\
		\vdots \\
		Y^\top F_k Y = m_k
	\end{cases}
	$$
	
	Si le système n'a pas de solution unique, nous générons un nouveau vinaigre et résolvons le système, jusqu'à ce que nous trouvions un système avec une seule solution.
	
	Soit $Y = \begin{bmatrix} O \\ V \end{bmatrix}$ avec $O, V \in \mathcal{F}_n^k$.
	
	Soit $F_e =
	\begin{bmatrix}
		0 & B_1 \\
		B_2 & B_3
	\end{bmatrix}$ avec $B_i \in \mathcal{F}_n^{k \times k}$.
	
	Soit $m_e \in \mathcal{F}_n$.
	
	Alors :
	\begin{align*}
		&&Y^\top F_e Y &= m_e \\
		&\Leftrightarrow &\begin{bmatrix} O^\top & V^\top \end{bmatrix} \begin{bmatrix}
			0 & B_1 \\
			B_2 & B_3
		\end{bmatrix} \begin{bmatrix} O \\ V \end{bmatrix} &= m_e \\
		&\Leftrightarrow &\begin{bmatrix} V^\top B_2 & (O^\top B_1 + V^\top B_3) \end{bmatrix} \begin{bmatrix} O \\ V \end{bmatrix} &= m_e \\
		&\Leftrightarrow &V^\top B_2 O + (O^\top B_1 + V^\top B_3) V &= m_e \\
		&\Leftrightarrow &V^\top B_2 O + V^\top B_1^\top O + V^\top B_3V &= m_e \\
		&\Leftrightarrow &(V^\top B_2 + V^\top B_1^\top) O &= m_e - V^\top B_3 V
	\end{align*}
	
	Et donc :
	
	$$
	\begin{cases}
		Y^\top F_1 Y = m_1 \\
		\vdots \\
		Y^\top F_k Y = m_k
	\end{cases}
	\Leftrightarrow\begin{bmatrix}
		V^\top B_{1, 2} + V^\top B_{1, 1}^\top \\
		\vdots \\
		V^\top B_{k, 2} + V^\top B_{k, 1}^\top
	\end{bmatrix}
	O = \begin{bmatrix}
		m_1 - V^\top B_{1, 3}V\\
		\vdots \\
		m_k - V^\top B_{k, 3}V
	\end{bmatrix}
	$$
	
	C'est un simple système $Ax = b$ à résoudre !
	
	Nous pouvons maintenant obtenir notre signature $X$ définie comme suit :
	
	$$X := A^{-1}Y$$
	
	\subsubsection{Verification}
	
	A signature $\mathbf{x} = (x_1,\dots,x_{2k})$ is valid if for all $i$, $x_i^TG_ix_i$ 
	
	\subsection{Proof of Correctness}
	
	salle u
	
	\subsection{Implementation}
	
	sait khi ?
	
	\section{An Improved Attack}
	
	\subsection{}
		
\end{document}